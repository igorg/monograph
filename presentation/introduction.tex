\begin{frame}
    \frametitle{Conceito e Exemplos}

    \begin{itemize}
        \item Rank significa uma posição particular, mais alta ou mais baixa que outras (dicionário Cambridge);
    
        \item Ranking é uma coleção dessas posições;
    
        \item Exemplos:

        \begin{itemize}
            \item Ranking das seleções de futebol masculino da FIFA;
            \item Ranking do IDH dos países (Índice de Desenvolvimento Humano);
            \item Ranking da competição da Yahoo! sobre Learning to Rank.
        \end{itemize}
    \end{itemize}
\end{frame}

\begin{frame}
    \frametitle{Características de um Ranking}

    \begin{itemize}
        \item Rankings são ordenados de acordo com critérios. Sobre esses critérios define-se um modelo, encarregado de impor uma ordem total ou parcial sobre os elementos do ranking;

        \item Características relevantes de um modelo:

        \begin{itemize}
            \item Computabilidade;
            \item Potencial de Automatização.
        \end{itemize}
    \end{itemize}
\end{frame}

\begin{frame}
    \frametitle{Problema Proposto}

    \begin{block}{Problema}
        Dado um conjunto de documentos em que cada documento pode receber um rótulo, atribuir uma posição a cada elemento do conjunto tendo como insumo pares documento-rótulo ou uma relação de ordem total ou parcial entre documentos;
    \end{block}

    \begin{itemize}
        \item Área de pesquisa chamada de Learning to Rank;
        \item Técnica de ranking reduzido a classificação é uma solução possível.
    \end{itemize}
\end{frame}