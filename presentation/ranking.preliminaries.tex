\begin{frame}
    \frametitle{Classificação e Ranking}

    \begin{description}
        \item[Classificação] Tarefa de atribuir rótulos a cada elemento de um dado conjunto tendo como entrada pares elemento-rótulo.

        \item[Ranking] Tarefa de atribuir posições a cada elemento de um dado conjunto tendo como entrada pares elemento-rótulo, uma relação de ordem parcial entre os elementos, ou uma relação de ordem total entre os elementos.
    \end{description}

    \begin{itemize}
        \item Algoritmos de classificação e ranking podem compartilhar o mesmo tipo de entrada, pares elemento-rótulo;
        \item Técnica de Ranking Reduzido a Classificação propõe compor um algoritmo de ranking a partir de um algoritmo de classificação.
    \end{itemize}
\end{frame}

\begin{frame}
    \frametitle{Descrição do Problema}

    \begin{block}{Problema}
        Dado um conjunto composto por elementos aos quais é possível atribuir uma classe de valor 0 ou 1, deseja-se encontrar uma permutação dos elementos de maneira que os elementos que apresentem maior chance de pertencer a classe 0 devem preceder os com maior chance de receber o rótulo 1.
    \end{block}

    \begin{itemize}
        \item Solução trivial, permutar os elementos do conjunto com base nas probabilidades das previsões de um classificador;
        \item Solução trivial pode implicar em erro alto na ordenação dependendo da performance do classificador;
        \item Técnica de Ranking Reduzido a Classificação é uma alternativa.
    \end{itemize}
\end{frame}

\begin{frame}
    \frametitle{Ranking Reduzido à Clasificação}

    \begin{itemize}
        \item Possui um limite teórico máximo de erro bastante reduzido se comparado à solução trivial;
        \item Dividido em etapas de treinamento e ordenação;
        \item Treinamento efetuado pelo algoritmo AUC-Train;
        \item Ordenação efetuada pelo algoritmo Tournament;
    \end{itemize}
\end{frame}

\begin{frame}
    \frametitle{Definições}
    
    \begin{block}{Base}
        Uma base $B$ é um conjunto de instâncias com cardinalidade $n$.
        \[
            B = \{i_1, ..., i_n\} \qquad |B| = n
        \]
    \end{block}

    \begin{block}{Instância}
        Uma instância I é uma tupla $\langle A, C \rangle$ na qual $A$ é um vetor de atributos com cardinalidade $m$ e $C$ é a classe da instância e pode valer 0 ou 1.
        \[
            I = \langle A, C \rangle \qquad A = (a_1, ..., a_m) \qquad C = x | x \in \{0, 1\}
        \]
    \end{block}

    \begin{itemize}
        \item Pode-se acessar os atributos de uma instância $I$ através de I(A) e classe através de I(C);
        \item Vetores de atributos podem ser concatenados através do operador binário $||$.
    \end{itemize}
\end{frame}

\begin{frame}
    \frametitle{Exemplo de Base}

    \begin{table}
        \centering
        \begin{tabular}{ccccc}
            \hline
            panorama & temperatura & humidade & ventoso & adequado \\
            \hline
            ensolarado & quente & alta & falso & não \\
            ensolarado & quente & alta & verdadeiro & não \\
            nublado & quente & alta & falso & sim \\
            chuvoso & branda & alta & falso & sim \\
            chuvoso & frio & normal & falso & sim \\
            chuvoso & frio & normal & verdadeiro & não \\
            nublado & frio & normal & verdadeiro & sim \\
            ensolarado & branda & alta & falso & não \\
            \hline
        \end{tabular}
    \end{table}
\end{frame}

\begin{frame}
    \frametitle{Medidas de Desempenho}

    \begin{block}{Acurácia}
        \begin{itemize}
            \item Razão entre o número de instâncias classificadas corretamente e o total de instâncias em uma base;
            \item Erros de classificação afetam pontualmente;
            \item Medida padrão para desempenho de classificadores.
        \end{itemize}
    \end{block}

    \begin{block}{AUC}
        \begin{itemize}
            \item Considera a relação entre as instâncias, em vez de uma simples razão como a acurácia;
            \item Erros de classificação podem afetar a AUC mais intensamente;
            \item Medida proposta para a avaliação de ranking.
        \end{itemize}
    \end{block}
\end{frame}