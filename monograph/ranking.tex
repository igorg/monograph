O algoritmo original de \emph{Ranking} descrito em \cite{langford} funciona em duas etapas. Na primeira, treina-se um classificador;
na segunda, o classificador gera o \emph{ranking} utilizando um método chamado \emph{torneio}. No algoritmo original, para cada etapa
o cenário de pior caso vale $O(n^2)$, um tempo computacional bastante elevado.

Duas abordagens foram consideradas para reduzir a complexidade do algoritmo: uma para a fase de treinamento e outra para a fase de geração do \emph{ranking}. Para a fase de classificação, foi pensada uma estratégia de votação entre classifiadores e para a geração do \emph{ranking}, foi pensada uma estratégia de torneio baseada em \emph{quicksort}. Nesse capítulo, estão explicados tanto o algoritmo original quanto essas abordagens para reduzir a complexidade.

\section{Descrição do algoritmo original}

\begin{definition}
Uma base $B$ é um conjunto de instâncias ${i_0, \dots i_n}$.
\end{definition}

\begin{definition}
Uma instância $I$ é composta por um conjunto de atributos $I(A) = {a_0, \dots a_m}$ e uma classe binária $I(C)$.
\end{definition}

\begin{definition}
Uma classe binária $C$ é um valor $c \in {0, 1}$.
\end{definition}

\begin{definition}
Um problema de classificação binária é definido como
\end{definition}

\begin{algorithm}
\begin{algorithmic}

\STATE train(S)
\STATE rank(S)

\caption{Algoritmo geral do \emph{Ranking}}
\label{alg_ranking}

\end{algorithmic}
\end{algorithm}

\begin{algorithm}
\begin{algorithmic}
    
\STATE $S$ \gets $\emptyset$
\FORALL{$I_0 \in S$}
    \FORALL{$I_1 \in S$}
        \IF{$I_0 \neq I_1$}
            \STATE $J(A)$ \gets $I_0(A) + I_1(A)$
            \STATE $J(C)$ \gets $1(I_0(C), I_1(C))$
        \ENDIF
    \ENDFOR
\ENDFOR

\caption{Etapa de treinamento segundo \cite{langford}}
\label{alg_original_train}

\end{algorithmic}    
\end{algorithm}

\begin{algorithm}
\begin{algorithmic}
    


\caption{Etapa de \emph{rank} segundo \cite{langford}}
\label{alg_original_train}

\end{algorithmic}    
\end{algorithm}

\section{Pares por instância}
Aqui ficará a explicação sobre a quantidade de pares por instância a treinar o classificador.

\section{Votação}
Aqui ficará a explicação sobre o sistema de votação entre vários classificadores.

\section{Quicksort}
Aqui ficará a explicação sobre quicksort para a geraçao do \emph{ranking}.