Como dito anteriormente, a técnica de \emph{ranking reduzido a classificação} aplica uma transformação à base original antes da etapa de treinamento do classificador e modifica a etapa de avaliação a fim de ordenar as instâncias.

A transformação consiste em formar novas instâncias a partir de pares entre instâncias de classes diferentes. A ordem em que as instâncias aparecem no par é relevante, assim, se existe um par $\langle i_\alpha, i_\beta \rangle$, deve existir também um par $\langle i_\beta, i_\alpha \rangle$. Além disso, as classes das duas instâncias que formam um par são combinadas a fim de gerar uma nova classe para a nova instância.

Essa transformação foi implantada separando a base de treinamento em dois subconjuntos, um subconjunto $S_0$ com as instâncias de classe $0$ e outro, $S_1$, com as instâncias de classe $1$, e efetuando o produto cartesiano entre os dois conjuntos da seguinte maneira $S_r = (S_0 \times S_1) \cup (S_1 \times S_0)$.

Após a aplicação do produto cartesiano, uma instância da base resultante tem a forma $\langle (A_\alpha, C_\alpha), (A_\beta, C_\beta) \rangle$. As classes das duas instâncias devem ser combinadas em uma nova classe, isso se dá com o uso do operador $1$. Ao fim desse processo, uma instancia da nova base terá forma $\langle (A_\alpha, A_\beta), 1 \cdot (C_\alpha < C_\beta)$.

O operador $1$, aplicado para obter a classe da nova instância, retorna $1$, se o resultado da expressão avaliada é verdadeiro, ou $0$, se o resultado da expressão é falsa. Como o argumento para o operador é $C_\alpha < C_\beta$, se $C_\alpha$ valer $1$, a classe da nova instância será $0$, se valer $0$, a classe da nova instância será $1$. A transformação da base é o treinamento do classificador compõe um algoritmo de nome \emph{AUC-Train} (algoritmo \ref{alg:auc-train}).

\begin{algorithm}
    \begin{algorithmic}

        \STATE Let $S' = \{\langle (x_1, x_2), 1 \cdot (y_1 < y_2) \rangle : (x_1, y_1), (x_2, y_2) \in S \wedge y_1 \neq y_2\}$
        \STATE return $c = A(S')$

        \caption{AUC-Train}
        \label{alg:auc-train}

    \end{algorithmic}
\end{algorithm}

A modificação da etapa de avaliação para gerar o \emph{ranking} da base de avaliação.

O algoritmo de torneio está transcrito no algoritmo \ref{alg:degree}.

\begin{algorithm}
    \begin{algorithmic}

        \FOR{$x \in U$}
            \STATE let $deg(x) = |\{x':c(x, x') = 1, x' \in U\}|$
        \ENDFOR

        \STATE Sort U in descending order of deg(x), breaking ties arbitrarily

        \caption{AUC-Train}
        \label{alg:degree}

    \end{algorithmic}
\end{algorithm}

O pior caso para o algoritmo \ref{alg:auc-train} acontece quando o conjunto de treinamento possui metade das instâncias com classe $0$ e a outra metade com classe $1$. O meio convencional de solucionar o produto cartesiano (algoritmo \ref{alg:cross-product}) é aninhar dois \emph{loops}. Supondo uma base com $n$ instâncias, o produto cartesiano ocorrerá em dois conjuntos com $n/2$ instâncias, resultando na complexidade assintótica $O(\frac{n}{2} \cdot \frac{n}{2}) = O(\frac{n^2}{4}) = O(n^2)$.

\begin{algorithm}
    \begin{algorithmic}
        \STATE $S \gets \emptyset$

        \FOR{$x \in S_1$}
            \FOR{$y \in S_2$}
                \STATE $S \gets S \cup \{(x, y)\}$
            \ENDFOR
        \ENDFOR

        \STATE returns S

        \caption{Cross Product}
        \label{alg:cross-product}

    \end{algorithmic}
\end{algorithm}

\emph{Não foi necessário se preocupar com os classificadores usados como base para o ordenador, essa é uma das vantagens do método em \cite{langford08}}.

O algoritmo proposto em é composto de duas etapas: uma de treinamento e outra de ordenação que gera o \emph{ranking}. Porém, o custo computacional desse algoritmo é alto em ambas etapas. Com bases de dados extensas o desempenho do algoritmo degrada consideravelmente, isso motivou a busca de alternativas para reduzir o tempo tanto de treinamento quanto de ordenação.

Duas abordagens foram consideradas para reduzir a complexidade do algoritmo: uma para a fase de treinamento e outra para a fase de geração do \emph{ranking}. Para a fase de classificação, foi pensada uma estratégia de votação entre classificadores e para a geração do \emph{ranking}, foi pensada uma estratégia de torneio baseada em \emph{quicksort}. Nesse capítulo, estão explicados tanto o algoritmo original quanto essas abordagens para reduzir a complexidade.

\section{Otimização do treinamento: Amostragem e Votação}
O algoritmo original chama a etapa de treinamento de AUC-Train. Nessa etapa, são feitas combinações entre as instâncias de classe $0$ e de classe $1$, como mostrado no \emph{algoritmo \ref{alg:auc-train}}.

Esse algoritmo de mesclagem das instâncias é executado uma vez e seu resultado é usado como massa de dados para um algoritmo de aprendizado; no caso específico dessa implantação, um classificador.

O custo total do AUC-Train é composto pelos custos da etapa de mesclagem e do algoritmo de aprendizagem somados. Vale ressaltar que a etapa de mesclagem tem um tempo maior de execução; a otimização endereça esse problema especificamente.

Na estratégia adotada para reduzir o tempo de mesclagem figuram duas técnicas conhecidas em \emph{data mining}, a primeira é efetuar uma amostragem do conjunto original e a segunda é promover uma votação entre vários classificadores treinados com as amostragens como massa de dados.

\begin{algorithm}
\begin{algorithmic}

\STATE Let $S_0 = \{\langle (x, y) \rangle: (x, y) \in S \wedge y = 0\}$
\STATE Let $S_1\,=\,S\,-\,S_0$

\STATE $V\,\gets\,\emptyset$

\FOR{$i\,=\,1,\;i\,\to\,iterations$}
\STATE Let $S_{sample} = S' : S' \subset S_1 \wedge |S'| = limit$
\STATE Let $S_{train} = \{\langle ((x_{0},\,x_{s}),\, 1),\; ((x_{s},\, x_{0}),\,0) \rangle:
       (x_{0},\,y_{0})\,\in\,S_{0}\,\wedge\,
       (x_{s},\,y_{s})\,\in\,S_{sample}\}$
\STATE $c\,\gets\,A(S_{train})$
\STATE $V\,\gets\,V\,\bigcup\,\{c\}$
\ENDFOR

\caption{AUC-Train com amostragem}
\label{alg:auc-train-amostragem-votacao}

\end{algorithmic}
\end{algorithm}

O produto gerado pelo algoritmo \ref{alg:auc-train-amostragem-votacao} é um conjunto de classificadores treinados a partir de amostragens do conjunto original. Se passarmos os parâmetros $limit=all$ e $iterations=1$, ele gera o mesmo resultado do algortimo descrito por {{langford}}.


\section{Otimização da ordenação: Quicksort}
A etapa de ordenação original é chamada de \emph{torneio}. Nessa abordagem, todas as instâncias do conjunto a ser ordenado são comparadas entre si com base no classificador obtido no treinamento e recebem uma pontuação. As instâncias de maior pontuação assumem as primeiras posições no \emph{ranking}. A proposta de otimização baseia-se em uma adaptação do algoritmo de quicksort para ordenar as instâncias do \emph{ranking}.

\begin{algorithm}
\begin{algorithmic}

\FORALL{$x\,\in\,U$}
\STATE $deg(x) = |\{x': c(x, x')=1, x'\in U\}|$
\ENDFOR
\STATE Ordene U de forma descendente com base em $deg(x)$, resolva os empates arbitrariamente

\caption{Degree}
\label{alg:degree}

\end{algorithmic}
\end{algorithm}


\section{Definições}
Algumas definições necessárias para o funcionamento do algoritmo final.

\section{Algoritmo final}

\begin{algorithm}
\begin{algorithmic}

\STATE $c \gets train(TS, LA, pairs, iterations)$
\STATE $rank(RS, c, quicksort)$

\caption{Algoritmo final do \emph{Ranking}}
\label{alg:ranking}

\end{algorithmic}
\end{algorithm}
