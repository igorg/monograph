O algoritmo proposto em {{langford08}} é composto de duas etapas: uma de treinamento e outra de ordenação que gera o \emph{ranking}. Porém, o custo computacional desse algoritmo é alto em ambas etapas. Com bases de dados extensas o desempenho do algoritmo degrada consideravelmente, isso motivou a busca de alternativas para reduzir o tempo tanto de treinamento quanto de ordenação.

Duas abordagens foram consideradas para reduzir a complexidade do algoritmo: uma para a fase de treinamento e outra para a fase de geração do \emph{ranking}. Para a fase de classificação, foi pensada uma estratégia de votação entre classificadores e para a geração do \emph{ranking}, foi pensada uma estratégia de torneio baseada em \emph{quicksort}. Nesse capítulo, estão explicados tanto o algoritmo original quanto essas abordagens para reduzir a complexidade.

\section{Otimização do treinamento: Votação e pares por instância}
O algoritmo original chama a etapa de treinamento de AUC-Train. Nessa etapa, são feitas combinações entre as instâncias de classe $0$ e de classe $1$, como mostrado no \emph{algoritmo \ref{alg:auc-train}}.

\begin{algorithm}
\begin{algorithmic}

\STATE Let $S' = {\langle (x_1, x_2), 1 \cdot (y_1 < y_2) \rangle : (x_1, y_1), (x_2, y_2) \in S \wedge y_1 \neq y_2}$
\STATE return $c = A(S')$

\caption{AUC-Train}
\label{alg:auc-train}

\end{algorithmic}
\end{algorithm}

O AUC-Train faz a combinação de cada instância com classe $0$ com todas as instâncias de classe $1$ nas formas $(z, u)$ e $(u, z)$, onde $z$ é a instância com classe $0$ e $u$ é a instância com classe $1$.

Para o pior caso, com um conjunto de treinamento em que metade das instâncias é da classe $0$ e a outra metade é da classe $1$, a complexidade assintótica é de $O(n^2)$.

Esse algoritmo de mesclagem das instâncias é executado uma vez e seu resultado é usado como massa de dados para um algoritmo de aprendizado; no caso específico dessa implantação, um classificador.

O custo total do AUC-Train é composto pelos custos da etapa de mesclagem e do algoritmo de aprendizagem somados. Vale ressaltar que a etapa de mesclagem tem um tempo maior de execução; a otimização endereça esse problema especificamente.

Na estratégia adotada para reduzir o tempo de mesclagem figuram duas técnicas conhecidas em \emph{data mining}, a primeira é efetuar uma amostragem do conjunto original e a segunda é promover uma votação entre vários classificadores treinados com as amostragens como massa de dados.

O algoritmo:

\begin{algorithm}
\begin{algorithmic}

\STATE Let $S_0 = {\langle (x, y) \rangle: (x, y) \in S \wedge y = 0}$
\STATE Let $S_1 = S - S_0$
\FOR{$i = 1$ \to iterations}
\STATE Let $S' = {{\langle (x_1, x_2), 1 \cdot (y_1 < y_2) \rangle : (x_1, y_1) \in S_0 \wedge (x_2, y_2) \in S_1}}$
\ENDFOR
\caption{AUC-Train com amostragem e votação}
\label{alg:auc-train-amostragem-votacao}

\end{algorithmic}
\end{algorithm}

\section{Otimização da ordenação: Quicksort}
A etapa de ordenação original é chamada de \emph{torneio}. Nessa abordagem, todas as instâncias do conjunto a ser ordenado são comparadas entre si com base no classificador obtido no treinamento e recebem uma pontuação. As instâncias de maior pontuação assumem as primeiras posições no \emph{ranking}. A proposta de otimização baseia-se em uma adaptação do algoritmo de quicksort para ordenar as instâncias do \emph{ranking}.

\section{Definições}
Algumas definições necessárias para o funcionamento do algoritmo final.

\section{Algoritmo final}

\begin{algorithm}
\begin{algorithmic}

\STATE $c \gets train(TS, LA, pairs, iterations)$
\STATE $rank(RS, c, quicksort)$

\caption{Algoritmo final do \emph{Ranking}}
\label{alg:ranking}

\end{algorithmic}
\end{algorithm}
