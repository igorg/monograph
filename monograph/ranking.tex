O algoritmo proposto em \cite{langford08} é composto de duas etapas: uma de treinamento e outra de ordenação que gera o \emph{ranking}. Porém, o custo computacional desse algoritmo é alto em ambas etapas, cada uma possui complexidade assintótica O(n^2). Isso nos levou a buscar alternativas para reduzir o tempo tanto de treinamento quanto de ordenação.

Na etapa de treinamento, o algoritmo original funciona fazendo todas as combinações entre as instâncias de classe $0$ e de classe $1$ existentes no conjunto de treinamento. A otimização pensada é aumentar o número de classificadores treinados e treiná-los com um subconjunto dos dados para treinamento.

\section{Otimização do treinamento: Votação e pares por instância}
Aqui ficará a explicação sobre o sistema de votação entre vários classificadores.

\section{Pares por instância}
Aqui ficará a explicação sobre a quantidade de pares por instância a treinar o classificador.

\section{Torneio}
