Segundo o dicionário Cambridge, a palavra \emph{rank} significa uma posição
particular, mais alta ou mais baixa que outras. Por exemplo, a seleção
brasileira de futebol masculino ocupa uma posição na classificação das
seleções da FIFA.

A posição ocupada pela seleção brasileira somada as posições ocupadas pelas
outras seleções constituem o \emph{ranking} das seleções de futebol
masculino da FIFA. Pode-se citar como outros exemplos de \emph{ranking}: o
\emph{ranking} das aplicações do Rails Rumble e o \emph{ranking} do IDH
(Índice de Desenvolvimento Humano) dos países.

Todos os \emph{rankings} citados são ordenados de acordo com um critério.
Para a definição da posição de uma seleção no \emph{ranking} de seleções
da FIFA, há um cálculo que leva em conta alguns fatores das partidas
disputadas por essa seleção nos quatro anos mais recentes. Exemplos:

\begin{itemize}
    \item Pontos por resultado (vitória, empate ou derrota);
    \item Importância da partida (amistoso, eliminatórias, copa do mundo, etc.);
    \item Força do adversário (posição no \emph{ranking}).
    \item Força da região (confederação FIFA)
    \item Período (Quanto mais recente, maior peso)
\end{itemize}

Com o cálculo feito para cada seleção vinculada à FIFA, define-se o
\emph{ranking} das seleções. Pode-se considerar que o topo do \emph{ranking}
possui as melhores seleções em um passado recente, enquanto na base estão
as piores seleções.