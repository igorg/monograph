Segundo o dicionário Cambridge, a palavra \emph{rank} significa uma posição particular, mais alta ou mais baixa que outras. Por exemplo, a seleção brasileira de futebol masculino ocupa uma posição na classificação das seleções da FIFA.

A posição ocupada pela seleção brasileira somada as posições ocupadas pelas outras seleções constituem o \emph{ranking} das seleções de futebol masculino da FIFA. Pode-se citar como outros exemplos de \emph{ranking}: o \emph{ranking} do IDH (Índice de Desenvolvimento Humano) dos países, o \emph{ranking} da competição da Yahoo! \footnote{http://learningtorankchallenge.yahoo.com/leaderboard.php} sobre \emph{Learning to rank}, entre outros.

Todos os \emph{rankings} citados são ordenados de acordo com um critério. Por exemplo, para a definição do \emph{ranking} de seleções da FIFA, calcula-se o total de pontos para cada seleção vinculada à FIFA e ordenam-se as seleções com base nas pontuações. As seleções com maior pontuação ocupam o topo do \emph{ranking}, enquanto as seleções com menor pontuação ocupam a base.

A pontuação total de uma seleção em um período de quatro anos é definida por meio da soma da média de pontos ganhos em partidas disputados nos 12 últimos meses e da média de pontos ganhos em partidas que ocorreram há mais de 12 meses, a qual se deprecia anualmente.

A pontuação para uma partida é calculada pela seguinte fórmula:

\[P = M \times I \times T \times C \times 100\]

Onde:
\begin{itemize}

    \item {\bf M}: Pontos por vitória (3 pontos), empate (1 ponto) ou derrota (0 pontos). Se a partida for decidida nos penaltis, o vencedor recebe 2 pontos e o derrotado recebe 1 ponto.

    \item {\bf I}: Importância da partida com os seguintes pesos:
    \begin{itemize}
        \item Amistoso: 1,0
        \item Eliminatórias da copa e eliminatórias continentais: 2,5
        \item Torneio continental e Copa da Confedarações: 3,0
        \item Copa do Mundo: 4,0
    \end{itemize}

    \item {\bf T}: Dificuldade do adversário de acordo com a seguinte fórmula:
    \[(200 - posicao \; do \; adversario \; no \; ranking) \div 100\]
    O time no topo do ranking reprensenta dificuldade 2,00. Os times abaixo da posicao 150 tem dificuldade fixa de 0,5.

    \item {\bf C}: Dificuldade de confederação: É a média entre as duas confederações envolvidas em uma partida. Cada confederação tem um peso, variando de 1 a 0.85, definido pela FIFA.
    
\end{itemize}

Podemos afirmar que as fórmulas utilizadas pela FIFA para definir as pontuações das seleções constituem um modelo que determina uma ordem parcial do conjunto de seleções FIFA. Dizer que o as fórmulas utilizadas pela FIFA constituem um modelo caracteriza a tarefa de ordenação das seleções FIFA como computável, ou seja, o modelo pode ser escrito em forma de algoritmo e avaliado, junto aos dados necessários, por um computador.

Outra característica do modelo FIFA é que ele foi definido por funcionários da instituição, ou seja, o modelo é fruto do intelecto humano. Apesar disso, seria possível a aprendizagem automática de um modelo que ordenasse o conjunto de seleções vinculadas à FIFA.

As características de computabilidade e possibilidade de aprendizagem automática, observadas no exemplo do \emph{ranking} da FIFA, podem ser encontradas em diversos outros tipos de ordenações.

Talvez não haja muita valia em automatizar o aprendizado de um modelo para ordenar as seleções vinculadas à FIFA; o modelo definido pela FIFA já é bastante adequado. Porém, para uma grande quantidade de aplicações, aprender modelos que ordenem uma base de dados pode ser útil, ajudando a lidar com problemas como quantidade de dados e variáveis a serem consideradas.

Um problema beneficiado pela aprendizagem de um modelo de ordenação seria: Dado um conjunto de documentos em que cada documento pode receber um rótulo, atribuir uma posicao a cada documento do conjunto, tendo como insumo pares documento-rótulo ou uma relação de ordem total ou parcial entre os documentos.

O problema de ordenação de documentos é um dos problemas de interesse de uma área  conhecida como \emph{Learning to Rank} que tem atraído a atenção de muitos pesquisadores nos últimos anos. Essa área de pesquisa é constituída por uma conjunção de conhecimentos sobre \emph{Aprendizagem de máquina} e \emph{Recuperação de informação} e está inserida no contexto de \emph{inteligência artificial}.

Essa área já conta com sites especializados no assunto\footnote{http://research.microsoft.com/en-us/um/beijing/projects/letor/}, competições\footnote{http://learningtorankchallenge.yahoo.com/} além do apoio das gigantes de TI.

Estudaremos a implantação e avaliação de um método de aprendizagem automática para ordenação de um conjunto de elementos passíveis de rotulação. O método consiste na aplicação de técnicas consolidadas em \emph{aprendizagem de máquina} (regressão e classificação) com finalidade de ordenar dados.