\emph{Ranking} pode ser enunciado, em uma forma simples, como a tarefa de
ordenar os elementos de um conjunto de acordo com algum critério. Essa é uma
definição bastante geral, porém estamos interessados em uma versão um pouco mais
restrita do problema.

Para nossos propósitos, os elementos do conjunto a ser ordenado são denominados
instâncias. Essas são compostas por um conjunto de atributos mais uma classe de
domínio ${0, 1}$.

Restringindo o problema, esperamos que o algoritmo de \emph{ranking} receba como
entrada um conjunto de instâncias no qual a classe esteja invisível durante o
processo de ordenação e retorne como saída o conjunto ordenado de forma que as
instâncias com classe $0$ devem preceder as com classe $1$.

Uma medida comum de sucesso para algoritmos de \emph{ranking} é a área sobre a
curva \emph{ROC (Receiver Operating Characteristic)}, comummente chamada de
\emph{AUC (Area Under the Curve)}. A perda associada a essa medida, $1 - AUC$,
reflete o número de instâncias, normalizado pela quantidade de $0$s vezes a
quantidade de $1$s, que necessitam ser trocadas para um \emph{ranking} perfeito.
Para uma ordenação perfeita, em que todas as instâncias com classe $0$ precedem
as com classe $1$, a perda na \emph{AUC} é $0$. No pior caso, em que todos os
$1$s precedem os $0$s, a perda na \emph{AUC} é $1$.

O uso da classe como critério para ordenação sugere uma relação com os problemas
de classificação e regressão em aprendizagem de máquina. Realmente, pode-se
derivar uma ordenação diretamente das pontuações obtidas em uma classificação ou
regressão. Se desejamos que instâncias da classe $0$ ocupem as primeiras
posições do \emph{ranking}, basta ordenar de forma crescente as pontuações
obtidas considerando a classe $0$.

Geralmente, a medida de eficiência mais utilizada para classificação é a
acurácia: uma razão entre o número de instâncias corretamente classificadas
sobre o número total de instâncias no conjunto de avaliação. O erro decorrente
de uma classificação afeta a acurácia de uma forma pontual.

Comparativamente, um erro de classificação pode ter maior influência na medida
\emph{AUC} que na acurácia, portanto afetar considerevelmente um \emph{ranking}.
A causa disso é a \emph{AUC} considerar uma relação entre as instâncias,
enquanto a acurácia considera apenas as instâncias pontualmente. Abaixo
ilustramos através de um exemplo uma relação entre essas medidas que comprova o
intuído sobre erros na classificação.

\begin{table}[h!]
        \centering
        \begin{tabular*}{0.50\textwidth}{@{\extracolsep{\fill}} c|rr}
        \hline
        atributos & classe & previsão \\
        \hline
        $a_{0, 1}, \quad \cdots, \quad a_{0, n}$ & 0 & 0 \\
        $a_{1, 1}, \quad \cdots, \quad a_{1, n}$ & 0 & 0 \\
        $a_{2, 1}, \quad \cdots, \quad a_{2, n}$ & 1 & 0 \\
        $a_{3, 1}, \quad \cdots, \quad a_{3, n}$ & 0 & 0 \\
        $a_{4, 1}, \quad \cdots, \quad a_{4, n}$ & 0 & 0 \\
        $a_{5, 1}, \quad \cdots, \quad a_{5, n}$ & 0 & 0 \\
        $a_{6, 1}, \quad \cdots, \quad a_{6, n}$ & 0 & 0 \\
        $a_{7, 1}, \quad \cdots, \quad a_{7, n}$ & 0 & 0 \\
        $a_{8, 1}, \quad \cdots, \quad a_{8, n}$ & 0 & 0 \\
        $a_{9, 1}, \quad \cdots, \quad a_{9, n}$ & 1 & 1 \\
        \hline
        \end{tabular*}

        \caption{Exemplo de \emph{ranking} e classificação}
\end{table}

No conjunto hipotético representado ao lado, temos dez instâncias ordenadas em
um \emph{ranking} com os atributos, as classes e as previsões dadas por um
classificador. Podemos perceber que o classificador errou apenas a classe da
terceira instância.

Calculando a acurácia, temos nove acertos em dez possíveis, ou seja, $90\%$.
Calculando a perda da AUC considerando como base a classe $0$, a terceira
instância precisa retroceder seis posições para uma ordenação perfeita,
normalizando pelo número de $0$s vezes o número de $1$s, temos
$(1 - AUC) = 6 \div (2 * 8) = 0,325$, logo a AUC vale $62,5\%$. Isso comprova
que a AUC, comparada à acurácia, pode sofrer um impacto maior devido a erros de
classificação.

Langford explica que um classificador que gere um erro de ordem $\alpha$ na
acurácia pode gerar um erro teórico máximo de $\alpha \cdot n$ na AUC, onde $n$
é a cardinalidade do conjunto de instâncias avaliado. Esse efeito se intensifica
a medida que aumenta o desbalanceamento de classes do conjunto usado durante o
processo de treinamento pois, quanto mais desbalanceadas as classes, mais
provável que o classificador resultante seja tendencioso para a classe
majoritária.
