\emph{Ranking} pode ser enunciado, em uma forma simples, como a tarefa de
ordenar os elementos de um conjunto de acordo com algum critério. Essa é uma
definição bastante geral, porém estamos interessados em uma versão um pouco mais
restrita do problema.

Para nossos propósitos, os elementos do conjunto a ser ordenado são denominados
instâncias. Essas são compostas por um conjunto de atributos mais uma classe de
domínio ${0, 1}$.

Restringindo o problema, esperamos que o algoritmo de \emph{ranking} receba como
entrada um conjunto de instâncias no qual a classe esteja invisível durante o
processo de ordenação e dê como saída o conjunto ordenado de forma que as
instâncias com classe $0$ devem preceder as com classe $1$.

A principal medida de eficiência da ordenação é a AUC
\emph{(Area Under the Curve)}

O uso da classe como critério para ordenação sugere uma relação com o problema
de classificação em aprendizagem de máquina. Na verdade, é possível derivar uma
ordem para as instâncias através das previsões produzidas por um classificador.

Se podemos derivar ordenações diretamente dos classificadores, por que existe
a necessidade de um algoritmo que seja especializado nisso? Segundo [langford],
a ordenação por um algoritmo especializado nesse problema nos fornece algumas
garantias que uma ordenação derivada de uma classificação não.

Falar de erro associado.

Antes de prosseguir, vamos definir formalmente algumas entidades necessárias
para a representação do problema de \emph{ranking}:

Um conjunto $S$ a ser ordenado é uma coleção de instâncias $i \in S$ em que toda
instância $i$ possui um espaço com $n$ características observáveis
$A_i = {a_1, ..., a_n}$ e uma classe $c_i \in {0, 1}$.

O arranjo das instâncias $i \in S$ na qual cada uma assume uma posição, ou
\emph{rank}, baseado em um critério é chamado \emph{ranking}. No nosso caso, o
arranjo é baseado na classe das instâncias.

A eficiência da ordenação do conjunto $S$ é máxima caso todas as instâncias
$i \in S$ tais que $c_i = 0$ estejam posicionadas antes de todas as instâncias
$i \in S$ tais que $c_i = 1$. Caso contrário, a eficiência é mínima.
