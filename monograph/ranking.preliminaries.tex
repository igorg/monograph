\emph{Ranking} pode ser enunciado, em uma forma simples, como a tarefa de
ordenar os elementos de um conjunto de acordo com algum crit�rio. Essa � uma
defini��o bastante geral, por�m estamos interessados em uma vers�o um pouco mais
restrita do problema.

Para nossos prop�sitos, os elementos do conjunto a ser ordenado s�o denominados
inst�ncias. Essas s�o compostas por um conjunto de atributos mais uma classe de
dom�nio ${0, 1}$.

Restringindo o problema, esperamos que o algoritmo de \emph{ranking} receba como
entrada um conjunto de inst�ncias no qual a classe esteja invis�vel durante o
processo de ordena��o e retorne como sa�da o conjunto ordenado de forma que as
inst�ncias com classe $0$ devem preceder as com classe $1$.

Uma medida comum de sucesso para algoritmos de \emph{ranking} � a �rea sobre a
curva ROC (\emph{(AUC - Area Under the Curve)}). A perda associada a essa medida
$1 - AUC$ reflete o n�mero de inst�ncias, normalizado pela quantidade de $0$s
vezes a quantidade de $1$s, que necessitam ser trocadas para um \emph{ranking}
perfeito. Para uma ordena��o perfeita, em que todas as inst�ncias com classe $0$
precedem as com classe $1$, a perda na \emph{AUC} � $0$. No pior caso, em que
todos os $1$s precedem os $0$s, a perda na \emph{AUC} � $1$.

O uso da classe como critério para ordenação sugere uma relação com o problema
de classificação em aprendizagem de máquina. Na verdade, é possível derivar uma
ordem para as instâncias através das previsões produzidas por um classificador.

Se podemos derivar ordenações diretamente dos classificadores, por que existe
a necessidade de um algoritmo que seja especializado nisso? Segundo [langford],
a ordenação por um algoritmo especializado nesse problema nos fornece algumas
garantias que uma ordenação derivada de uma classificação não.

Falar de erro associado.

Antes de prosseguir, vamos definir formalmente algumas entidades necessárias
para a representação do problema de \emph{ranking}:

Um conjunto $S$ a ser ordenado é uma coleção de instâncias $i \in S$ em que toda
instância $i$ possui um espaço com $n$ características observáveis
$A_i = {a_1, ..., a_n}$ e uma classe $c_i \in {0, 1}$.

O arranjo das instâncias $i \in S$ na qual cada uma assume uma posição, ou
\emph{rank}, baseado em um critério é chamado \emph{ranking}. No nosso caso, o
arranjo é baseado na classe das instâncias.

A eficiência da ordenação do conjunto $S$ é máxima caso todas as instâncias
$i \in S$ tais que $c_i = 0$ estejam posicionadas antes de todas as instâncias
$i \in S$ tais que $c_i = 1$. Caso contrário, a eficiência é mínima.
