A primeira conclusão que pode ser tirada é que o algoritmo testado aqui possui tempo computacional elevado tanto para treinamento, quanto para avaliação. As estratégias para melhoria no tempo de treinamento, como uso de votação e de um número reduzido de pares por instância, ajudaram nesse aspecto e criaram, por vezes, ordenações melhores que as criadas pelo algoritmo original e pelo classificador base.

O classificador Naïve Bayes teve um desempenho incomum quando combinado com o algortimo de \emph{Ranking}. Esse classificador gerou resultados muito abaixo do esperado para todas as bases testadas, exceto para a base \emph{yeast}. Esse comportamento anômalo não foi elucidado nesse estudo.

Comparando as otimizações propostas para acelerar a etapa de treinamento de forma isolada, pode-se chegar a seguinte conclusão: o aumento do número de classificadores na votação cria resultados com menor variação na AUC que o aumento de pares por instância no treinamento.

Com a exceção do classificador naïve-bayes, que apresentou um comportamento anômalo, a técnica avaliada resultou em menor perda na AUC em comparação com a ordenação do classificador solo na maioria dos casos.

Para a árvore de decisão C4.5 ordenando a base Yeast --- a mais desbalanceada de todas as bases testadas --- a técnica apresentou uma melhoria considerável em relação à ordenação da árvore de decisão solo. Esse é um caso que tende a ratificar o ponto exposto em \cite{langford08}.

Em quase todos os casos, a otimização de votação elevou a performance da técnica original de forma que os resultados obtidos fossem melhores que os resultados dos classificadores solo, e algumas vezes superiores à técnica de \emph{ranking reduzido a classificação}.