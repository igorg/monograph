\documentclass[12pt, a4paper]{abnt}
\usepackage[brazilian]{babel}
\usepackage[utf8]{inputenc}

\begin{document}

\begin{resumo}
O estudo contemporâneo da inteligência artificial apresenta várias áreas de conhecimento, dentre essas, a aprendizagem de máquina, que busca entender o aprendizado como um processo computacional.
\emph{Ranking} é um dos assuntos nesse campo que vem ganhando maior atenção nos últimos anos.
Nesse trabalho, implantaremos uma solução alternativa para o problema de \emph{ranking}: um meta-classificador que utiliza como base um classificador binário - um dos algoritmos mais estudados em aprendizagem de máquina - e avaliaremos seu desempenho.
O objetivo é estabelecer uma ordem das instâncias em uma base de dados pela provável classe de cada instância.
O framework utilizado para a implantação e avaliação do meta-classificador é o \emph{Waikato Environment for Knowledge Analisys - WEKA} da universidade de Waikato, Nova Zelândia.
\end{resumo}

\begin{abstract}
The contemporary study of artificial intelligence has several areas of knowledge, among these, machine learning, which seeks to understand learning as a computational process.
\emph{Ranking} is an issue in this field that is gaining increasing attention in recent years.
In this work, will implement an alternative solution to the problem of \emph{ranking}: a meta-classifier using a classifier based on binary - one of the most studied algorithms in machine learning - and evaluate their performance.
The goal is to establish an order of the instances in a database for the likely class for each instance. The framework used for the implementation and evaluation of the meta-classifier is the \emph{Waikato Environment for Knowledge Analisys - WEKA} University of Waikato, New Zealand.
\end{abstract}

Segundo o dicionário Cambridge, a palavra \emph{rank} significa uma posição
particular, mais alta ou mais baixa que outras. Por exemplo, a seleção
brasileira de futebol masculino ocupa uma posição na classificação das
seleções da FIFA.

A posição ocupada pela seleção brasileira somada as posições ocupadas pelas
outras seleções constituem o \emph{ranking} das seleções de futebol
masculino da FIFA. Pode-se citar como outros exemplos de \emph{ranking} o
\emph{ranking} das aplicações do Rails Rumble e o \emph{ranking} do IDH
(Índice de Desenvolvimento Humano) dos países.

Todos os \emph{rankings} citados são ordenados de acordo com um critério.
Para a definição da posição de uma seleção no \emph{ranking} de seleções
da FIFA, há um cálculo que leva em conta os alguns fatores das partidas
disputadas por essa seleção nos quatro anos mais recentes. Exemplos:

\begin{list}
    \item Pontos por resultado (vitória, empate ou derrota);
    \item Importância da partida (amistoso a copa do mundo);
    \item Força do adversário (posição no \emph{ranking}).
    \item Força da região (confederação FIFA)
    \item Período (Mais recente, maior peso)
\end{list}

Com o cálculo feito para cada seleção vinculada à FIFA, define-se o
\emph{ranking} das seleções. Pode-se considerar que o topo do \emph{ranking}
possui as melhores seleções em um passado recente, enquanto na base estão
as piores seleções.

Em termos gerais, \emph{ranking} pode ser enunciado como a tarefa de
ordenar os elementos de um conjunto de acordo com algum critério.

Todos os exemplos citados acima compartilham um aspecto comum: a ordem dos
elementos do ranking é feita sobre uma pontuação de cada elemento obtida a
partir de algumas características. Da mesma forma, o problema a ser estudado
consiste em ordenar elementos de acordo com uma pontuação.

Dado um conjunto de instâncias, espera-se obter um arranjo desse que considere
alguma característica de tais instâncias.

Pode-se descrever o problema como a busca pela ordenação dos elementos de um
conjunto. A ordem é regida por uma pontuação obtida por cada elemento, assim
podemos reduzir o problema a determinação da pontuação para todos os
elementos do conjunto.

\end{document}
