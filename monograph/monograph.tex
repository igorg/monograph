\documentclass[12pt, a4paper, normaltoc, capchap, capsec, times]{abnt}

\usepackage[brazilian]{babel}
\usepackage[utf8]{inputenc}
\usepackage{wrapfig}
\usepackage{multirow}
\usepackage{amsthm}
\usepackage{amsfonts}
\usepackage{amsmath}
\usepackage[ruled,lined,portuguese]{algorithm2e}
\usepackage{graphicx}
\usepackage[config]{subfig}
\usepackage{abntcite}
\usepackage{uff}

\begin{document}

\looseness+1

\begin{resumo}
Essa monografia tem por objetivo implantar e avaliar a técnica de \emph{ranking reduzido a classificação binária} descrita no artigo \cite{langford08}. Tal técnica pertence à área de estudo chamada \emph{Learning to Rank} inserida na macroárea de Aprendizado de Máquina no campo de Inteligência Artificial. O problema emerge de situações reais como: ordenar documentos de acordo com uma busca e escolha de produtos a serem recomendados em um \emph{e-commerce}.
\end{resumo}

\begin{abstract}
This monograph has for goal to implement and evaluate a reduction from ranking to classification as proposed in \cite{langford08}. Such technique belongs to an area named Learning to Rank, inserted in Machine Learning branch in the field of Artificial Inteligence. The problem arises from real situations as: document ordering according to some query and selection of products to recommend in a e-commerce.
\end{abstract}

\chapter{Introdução}
\label{chap:introducao}
Segundo o dicionário Cambridge, a palavra \emph{rank} significa uma posição
particular, mais alta ou mais baixa que outras. Por exemplo, a seleção
brasileira de futebol masculino ocupa uma posição na classificação das
seleções da FIFA.

A posição ocupada pela seleção brasileira somada as posições ocupadas pelas
outras seleções constituem o \emph{ranking} das seleções de futebol
masculino da FIFA. Pode-se citar como outros exemplos de \emph{ranking} o
\emph{ranking} das aplicações do Rails Rumble e o \emph{ranking} do IDH
(Índice de Desenvolvimento Humano) dos países.

Todos os \emph{rankings} citados são ordenados de acordo com um critério.
Para a definição da posição de uma seleção no \emph{ranking} de seleções
da FIFA, há um cálculo que leva em conta os alguns fatores das partidas
disputadas por essa seleção nos quatro anos mais recentes. Exemplos:

\begin{list}
    \item Pontos por resultado (vitória, empate ou derrota);
    \item Importância da partida (amistoso a copa do mundo);
    \item Força do adversário (posição no \emph{ranking}).
    \item Força da região (confederação FIFA)
    \item Período (Mais recente, maior peso)
\end{list}

Com o cálculo feito para cada seleção vinculada à FIFA, define-se o
\emph{ranking} das seleções. Pode-se considerar que o topo do \emph{ranking}
possui as melhores seleções em um passado recente, enquanto na base estão
as piores seleções.

Em termos gerais, \emph{ranking} pode ser enunciado como a tarefa de
ordenar os elementos de um conjunto de acordo com algum critério.

Todos os exemplos citados acima compartilham um aspecto comum: a ordem dos
elementos do ranking é feita sobre uma pontuação de cada elemento obtida a
partir de algumas características. Da mesma forma, o problema a ser estudado
consiste em ordenar elementos de acordo com uma pontuação.

Dado um conjunto de instâncias, espera-se obter um arranjo desse que considere
alguma característica de tais instâncias.

Pode-se descrever o problema como a busca pela ordenação dos elementos de um
conjunto. A ordem é regida por uma pontuação obtida por cada elemento, assim
podemos reduzir o problema a determinação da pontuação para todos os
elementos do conjunto.

\chapter{\emph{Ranking}: Noções Preliminares}
\label{chap:nocoes_preliminares}
\emph{Ranking} pode ser enunciado, em uma forma simples, como a tarefa de
ordenar os elementos de um conjunto de acordo com algum critério. Essa é uma
definição bastante geral, porém estamos interessados em uma versão um pouco mais
restrita do problema.

Para nossos propósitos, os elementos do conjunto a ser ordenado são denominados
instâncias. Essas são compostas por um conjunto de atributos mais uma classe de
domínio ${0, 1}$.

Restringindo o problema, esperamos que o algoritmo de \emph{ranking} receba como
entrada um conjunto de instâncias no qual a classe esteja invisível durante o
processo de ordenação e retorne como saída o conjunto ordenado de forma que as
instâncias com classe $0$ devem preceder as com classe $1$.

Uma medida comum de sucesso para algoritmos de \emph{ranking} é a área sobre a
curva \emph{ROC (Receiver Operating Characteristic)}, comummente chamada de
\emph{AUC (Area Under the Curve)}. A perda associada a essa medida, $1 - AUC$,
reflete o número de instâncias, normalizado pela quantidade de $0$s vezes a
quantidade de $1$s, que necessitam ser trocadas para um \emph{ranking} perfeito.
Para uma ordenação perfeita, em que todas as instâncias com classe $0$ precedem
as com classe $1$, a perda na \emph{AUC} é $0$. No pior caso, em que todos os
$1$s precedem os $0$s, a perda na \emph{AUC} é $1$.

O uso da classe como critério para ordenação sugere uma relação com os problemas
de classificação e regressão em aprendizagem de máquina. Realmente, pode-se
derivar uma ordenação diretamente das pontuações obtidas em uma classificação ou
regressão. Se desejamos que instâncias da classe $0$ ocupem as primeiras
posições do \emph{ranking}, basta ordenar de forma crescente as pontuações
obtidas considerando a classe $0$.

Geralmente, a medida de eficiência mais utilizada para classificação é a
acurácia: uma razão entre o número de instâncias corretamente classificadas
sobre o número total de instâncias no conjunto de avaliação. O erro decorrente
de uma classificação afeta a acurácia de uma forma pontual.

Comparativamente, um erro de classificação pode ter maior influência na medida
\emph{AUC} que na acurácia, portanto afetar considerevelmente um \emph{ranking}.
A causa disso é a \emph{AUC} considerar uma relação entre as instâncias,
enquanto a acurácia considera apenas as instâncias pontualmente. Abaixo
ilustramos através de um exemplo uma relação entre essas medidas que comprova o
intuído sobre erros na classificação.

\begin{table}[h!]
        \centering
        \begin{tabular*}{0.50\textwidth}{@{\extracolsep{\fill}} c|rr}
        \hline
        atributos & classe & previsão \\
        \hline
        $a_{0, 1}, \quad \cdots, \quad a_{0, n}$ & 0 & 0 \\
        $a_{1, 1}, \quad \cdots, \quad a_{1, n}$ & 0 & 0 \\
        $a_{2, 1}, \quad \cdots, \quad a_{2, n}$ & 1 & 0 \\
        $a_{3, 1}, \quad \cdots, \quad a_{3, n}$ & 0 & 0 \\
        $a_{4, 1}, \quad \cdots, \quad a_{4, n}$ & 0 & 0 \\
        $a_{5, 1}, \quad \cdots, \quad a_{5, n}$ & 0 & 0 \\
        $a_{6, 1}, \quad \cdots, \quad a_{6, n}$ & 0 & 0 \\
        $a_{7, 1}, \quad \cdots, \quad a_{7, n}$ & 0 & 0 \\
        $a_{8, 1}, \quad \cdots, \quad a_{8, n}$ & 0 & 0 \\
        $a_{9, 1}, \quad \cdots, \quad a_{9, n}$ & 1 & 1 \\
        \hline
        \end{tabular*}

        \caption{Exemplo de \emph{ranking} e classificação}
\end{table}

No conjunto hipotético representado ao lado, temos dez instâncias ordenadas em
um \emph{ranking} com os atributos, as classes e as previsões dadas por um
classificador. Podemos perceber que o classificador errou apenas a classe da
terceira instância.

Calculando a acurácia, temos nove acertos em dez possíveis, ou seja, $90\%$.
Calculando a perda da AUC considerando como base a classe $0$, a terceira
instância precisa retroceder seis posições para uma ordenação perfeita,
normalizando pelo número de $0$s vezes o número de $1$s, temos
$(1 - AUC) = 6 \div (2 * 8) = 0,325$, logo a AUC vale $62,5\%$. Isso comprova
que a AUC, comparada à acurácia, pode sofrer um impacto maior devido a erros de
classificação.

Langford explica que um classificador que gere um erro de ordem $\alpha$ na
acurácia pode gerar um erro teórico máximo de $\alpha \cdot n$ na AUC, onde $n$
é a cardinalidade do conjunto de instâncias avaliado. Esse efeito se intensifica
a medida que aumenta o desbalanceamento de classes do conjunto usado durante o
processo de treinamento pois, quanto mais desbalanceadas as classes, mais
provável que o classificador resultante seja tendencioso para a classe
majoritária.


\chapter{\emph{Ranking}: Implantação}
\label{chap:implantacao}
O algoritmo proposto em \cite{langford08} é composto de duas etapas: uma de treinamento e outra de ordenação que gera o \emph{ranking}. Porém, o custo computacional desse algoritmo é alto em ambas etapas, cada uma possui complexidade assintótica O(n^2). Isso nos levou a buscar alternativas para reduzir o tempo tanto de treinamento quanto de ordenação.

Na etapa de treinamento, o algoritmo original funciona fazendo todas as combinações entre as instâncias de classe $0$ e de classe $1$ existentes no conjunto de treinamento. A otimização pensada é aumentar o número de classificadores treinados e treiná-los com um subconjunto dos dados para treinamento.

O algoritmo original chama a etapa de ordenação de \emph{torneio}. Nessa abordagem, todas as instâncias do conjunto a ser ordenado são comparadas entre si com base no classificador obtido no treinamento e recebem uma pontuação. As instâncias de maior pontuação assumem as primeiras posições no \emph{ranking}. A proposta de otimização baseia-se em uma adaptação do algoritmo de quicksort para ordenar as instâncias do \emph{ranking}.

\section{Otimização do treinamento: Votação e pares por instância}
Aqui ficará a explicação sobre o sistema de votação entre vários classificadores.

\section{Otimização da ordenação: Quicksort}
Aqui ficará a explicação sobre a quantidade de pares por instância a treinar o classificador.

\section{Algoritmo final}


\chapter{Avaliação do \emph{Ranking}}
\label{chap:avaliacao}
\begin{frame}
    \frametitle{Premissas}

    \begin{itemize}
        \item Um classificador que apresente erro de $\alpha$ na acurácia, pode apresentar um erro máximo de $\alpha \cdot n$ na ordenação onde $n$ é o tamanho da base a ser ordenada;
        \item Para o mesmo classificador com erro de $\alpha$ na acurácia, a técnica de Ranking Reduzido a Classificação reduz o erro máximo na ordenação para $\alpha \cdot 2$;
        \item Quanto maior o desbalanceamento entre as classes em uma base, maior a chance de intensificação do erro na ordenação.
    \end{itemize}
\end{frame}

\begin{frame}
    \frametitle{Características das Bases}
    
    \begin{block}{Observações}
        \begin{itemize}
            \item Foram escolhidas bases com diferentes níveis de desbalanceamento a fim de verificar as premissas.
            \item Algumas bases que tratavam originalmente de problemas multiclasse precisaram ser convertidas para bases binárias.
        \end{itemize}
    \end{block}

    \begin{table}[H]
    
        \begin{tabular}{c c c c}
            \hline
            \multirow{2}{*}{Bases} & \multicolumn{3}{c}{Classe} \\ \cline{2-4}
            & {\small Minoritária} & {\small Majoritária} & {\small Distribuição}\\
            \hline
            breast-cancer & 85 & 201 & 30\% - 70\%\\
            vehicle & 199 & 647 & 23\% - 77\%\\
            hepatitis & 32 & 123 & 20\% - 80\%\\
            glass & 29 & 185 & 13\% - 87\%\\
            yeast & 20 & 463 & 4\% - 96\%\\
            \hline
        \end{tabular}
    
        \caption{Dados sobre as bases usadas para \emph{ranking}}
    \end{table}
\end{frame}

\begin{frame}
    \frametitle{Classificadores Avaliados}
    
    \begin{itemize}
        \item Árvore de Decisão C4.5 (trees.J48)
        \item Naïve Bayes (bayes.NaiveBayes)
        \item Curva Logística (functions.Logistic)
        \item Support Vector Machine (functions.SMO)
    \end{itemize}
\end{frame}

\begin{frame}
    \frametitle{Estratégia de avaliação}

    Os testes executaram através de validação cruzada com 10 partições nas seguintes configurações:

    \begin{enumerate}
        \item Somente o classificador;
        \item O classificador como base para a técnica de \emph{ranking reduzido a classificação} original;
        \item O classificador como base para o algoritmo de ranking com configurações de 1 par por instância e variando o número de classificadores na votação entre 1 e 20;
        \item O classificador como base para o algoritmo de ranking com configurações de 1 classificador na votação e variando o número de pares por instância entre 1 e 20.
    \end{enumerate}
\end{frame}

\begin{frame}
    \frametitle{Desempenho: Árvore de decisão C4.5 (trees.J48)}

    \begin{figure}[H]
        \centering
        \includegraphics[width=0.9\textwidth]{img/yeast_j48.eps}
        \caption{Gráfico de desempenho para a base Yeast}
    \end{figure}
\end{frame}

\begin{frame}
    \frametitle{Desempenho: Naïve Bayes (bayes.NaiveBayes)}

    \begin{figure}[H]
        \centering
        \includegraphics[width=0.9\textwidth]{img/breast-cancer_naive-bayes.eps}
        \caption{Gráfico de desempenho para a base Breast Cancer}
    \end{figure}
\end{frame}

\begin{frame}
    \frametitle{Desempenho: Naïve Bayes (bayes.NaiveBayes)}

    \begin{figure}[H]
        \centering
        \includegraphics[width=0.9\textwidth]{img/glass_naive-bayes.eps}
        \caption{Gráfico de desempenho para a base Glass}
    \end{figure}
\end{frame}

\begin{frame}
    \frametitle{Desempenho: Naïve Bayes (bayes.NaiveBayes)}

    \begin{figure}[H]
        \centering
        \includegraphics[width=0.9\textwidth]{img/yeast_naive-bayes.eps}
        \caption{Gráfico de desempenho para a base Yeast}
    \end{figure}
\end{frame}


\begin{frame}
    \frametitle{Desempenho: Curva Logística (functions.Logistic)}

    \begin{figure}[H]
        \centering
        \includegraphics[width=0.9\textwidth]{img/yeast_logistic.eps}
        \caption{Gráfico de desempenho para a base Yeast}
    \end{figure}
\end{frame}

\begin{frame}
    \frametitle{Desempenho: Support Vector Machine (functions.SMO)}

    \begin{figure}[H]
        \centering
        \includegraphics[width=0.9\textwidth]{img/vehicle_smo.eps}
        \caption{Gráficos de desempenho para a base Vehicle}
    \end{figure}
\end{frame}

\begin{frame}
    \frametitle{Desempenho: Support Vector Machine (functions.SMO)}

    \begin{figure}[H]
        \centering
        \includegraphics[width=0.9\textwidth]{img/yeast_smo.eps}
        \caption{Gráficos de desempenho para a base Vehicle}
    \end{figure}
\end{frame}

\chapter{Conclusão}
\label{chap:conclusao}
A primeira conclusão que pode ser tirada é que o algoritmo testado aqui possui tempo computacional elevado tanto para treinamento, quanto para avaliação. As estratégias para melhoria no tempo de treinamento, como uso de votação e de um número reduzido de pares por instância, ajudaram nesse aspecto e criaram resultados por vezes superiores ao algoritmo original e ao classificador base.

Os esforços para otimizar o tempo na etapa de avaliação consistiram em implantar um algoritmo baseado no algoritmo \emph{Quicksort}, que teria um limite superior de chamadas à \emph{função \ref{func:votacao}} de $O(n \cdot \log(n))$, um avanço comparado ao limite superior do torneio que é $O(n^2)$. Embora esta estratégia tenha sido completamente implantada, não houve nenhum experimento que a utilizasse.

O classificador Naïve Bayes teve um desempenho incomum quando combinado com o algortimo de \emph{Ranking}. Esse classificador gerou resultados absurdamente abaixo do esperado para todas as bases testadas, exceto para a base \emph{yeast}.

Analisando os gráficos da seçao \ref{sec:avaliacao}, repara-se que, para a base \emph{glass}, os resultados começam a melhorar em relação aos obtidos para as bases \emph{breast-cancer}, \emph{vehicle} e \emph{hepatitis}. Já para a base \emph{yeast}, o Naïve Bayes aliado ao algoritmo de \emph{Ranking} teve uma performance excelente, acertando quase todas as ordenações. Esse comportamento anômalo não foi elucidado nesse estudo.

Comparando as estratégias implantadas para acelerar a etapa de treinamento de forma isolada, pode-se chegar a seguinte conclusão: o aumento do número de classificadores na votação cria resultados com menor variação na AUC que o aumento de pares por instância no treinamento.

Olhando as curvas geradas para cada uma dessas estratégias nos gráficos do capítulo \ref{chap:avaliacao}, percebe-se que a tendência da AUC para a estratégia de aumento de classificadores na votação é, na maioria dos casos, crescente. Enquanto a tendência da AUC para a estratégia de aumento do número de pares por instância não pode ser definida com clareza em alguns casos.

O artigo \cite{langford08} mostra que um classificador que produza um erro $\alpha$ na acurácia pode produzir um erro teórico máximo de $n \cdot \alpha$ na AUC, onde $n$ é o número de exemplos a serem ordenados. Mostra também que a técnica de \emph{ranking reduzido a classificação} produz um erro teórico máximo de $2 \cdot \alpha$.

Com a exceção do classificador naïve-bayes, que apresentou um comportamento anômalo, a técnica avaliada resultou em menor perda na AUC em comparação com a ordenação do classificador solo na maioria dos casos.

Com o uso da árvore de decisão C4.5 ordenando a base Yeast --- a mais desbalanceada de todas as bases testadas --- a técnica apresentou um incremento de aproximadamente $100\%$ comparada à ordenação da árvore de decisão solo. Esse é um caso que tende a ratificar o ponto exposto em \cite{langford08}.

Em quase todos os casos, a otimização de votação elevou a performance da técnica original de forma que os resultados obtidos fossem melhores que os resultados dos classificadores solo, e muitas vezes superiores à técnica de \emph{ranking reduzido a classificação}.

\bibliography{monograph}

\end{document}
