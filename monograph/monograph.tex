\documentclass[12pt, a4paper]{abnt}
\usepackage[brazilian]{babel}
\usepackage[utf8]{inputenc}

\begin{document}

\begin{resumo}
O estudo contemporâneo da inteligência artificial apresenta várias áreas de conhecimento, dentre essas, a aprendizagem de máquina, que busca entender o aprendizado como um processo computacional.
\emph{Ranking} é um dos assuntos nesse campo que vem ganhando maior atenção nos últimos anos.
Nesse trabalho, implantaremos uma solução alternativa para o problema de \emph{ranking}: um meta-classificador que utiliza como base um classificador binário - um dos algoritmos mais estudados em aprendizagem de máquina - e avaliaremos seu desempenho.
O objetivo é estabelecer uma ordem das instâncias em uma base de dados pela provável classe de cada instância.
O framework utilizado para a implantação e avaliação do meta-classificador é o \emph{Waikato Environment for Knowledge Analisys - WEKA} da universidade de Waikato, Nova Zelândia.
\end{resumo}

\begin{abstract}
The contemporary study of artificial intelligence has several areas of knowledge, among these, machine learning, which seeks to understand learning as a computational process.
\emph{Ranking} is an issue in this field that is gaining increasing attention in recent years.
In this work, will implement an alternative solution to the problem of \emph{ranking}: a meta-classifier using a classifier based on binary - one of the most studied algorithms in machine learning - and evaluate their performance.
The goal is to establish an order of the instances in a database for the likely class for each instance. The framework used for the implementation and evaluation of the meta-classifier is the \emph{Waikato Environment for Knowledge Analisys - WEKA} University of Waikato, New Zealand.
\end{abstract}

\begin{frame}
    Introdução
\end{frame}

\end{document}
